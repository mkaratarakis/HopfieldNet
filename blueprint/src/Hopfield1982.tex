\chapter{Hopfield's 1982 paper}

This module formalizes the key mathematical concepts from J.J. Hopfield's 1982 paper \cite{hopfield1982neural}
"Neural networks and physical systems with emergent collective computational abilities",
focusing on aspects that are not already covered in the main HopfieldNet formalization.

The paper introduces a model of neural networks with binary neurons and studies their collective
computational properties, particularly as content-addressable memories. The key insights include:

\begin{itemize}
\item The phase space flow and stable states of the network
\item The storage capacity and pattern retrieval capabilities
\item The relationship between energy minimization and memory retrieval
\item Tolerance to noise and component failures
\end{itemize}

\begin{definition}\label{PhaseSpacePoint}
\lean{Hopfield82.PhaseSpacePoint}
\uses{HopfieldNetwork}
\leanok
A `PhaseSpacePoint` represents a state in the phase space of the Hopfield system.
In the paper, this corresponds to the instantaneous state of all neurons (p.2554):
"A point in state space then represents the instantaneous condition of the system."
\end{definition}


\begin{definition}\label{localField}
\lean{Hopfield82.localField}
\uses{HopfieldNetwork, PhaseSpacePoint}
\leanok
The `localField` for neuron i in state s is the weighted sum of inputs from other neurons,
minus the threshold. This corresponds to $\sum j Tij Vj - \theta_i$ in the paper.
\end{definition}


\begin{definition}\label{updateRule}
\lean{Hopfield82.updateRule}
\uses{HopfieldNetwork, PhaseSpacePoint}
\leanok
The `updateRule` defines the neural state update according to the paper's Equation 1:
    $Vi \rightarrow 1 if \sum j Tij Vj > Ui$
   $ Vi \rightarrow 0 if \sum j Tij Vj < Ui$
In our formalization, we use -1 instead of 0 for the "not firing" state.
\end{definition}

\begin{definition}\label{PhaseSpaceFlow}
\lean{Hopfield82.PhaseSpaceFlow}
\uses{PhaseSpacePoint}
\leanok
A `PhaseSpaceFlow` describes how the system state evolves over time.
It maps each point in phase space to its successor state after updating one neuron.
From the paper (p.2554): "The equations of motion of the system describe a flow in state space."
\end{definition}


\begin{definition}\label{FixedPoint}
\lean{Hopfield82.FixedPoint}
\uses{PhaseSpacePoint}
\leanok
A `FixedPoint` of the phase space flow is a state that does not change under evolution.
In the paper, these correspond to the locally stable states of the network (p.2554):
"Various classes of flow patterns are possible, but the systems of use for memory
particularly include those that flow toward locally stable points..."
\end{definition}


\begin{definition}\label{BasinOfAttraction}
\lean{Hopfield82.BasinOfAttraction}
\uses{FixedPoint}
\leanok
A `BasinOfAttraction` of a fixed point is the set of all states that converge to it.
In the paper (p.2554): "Then, if the system is started sufficiently near any Xa,
as at $X = Xa + \Delta,$ it will proceed in time until $X \equiv Xa.$"
\end{definition}


\begin{definition}\label{EnergyLandscape}
\lean{Hopfield82.EnergyLandscape}
\uses{HopfieldNetwork}
\leanok
The `EnergyLandscape` of a Hopfield network is the energy function defined over all possible states.
In the paper, this is the function E defined in Equation 7:
   $ E = -1/2 \sum \sum Tij Vi Vj$
\end{definition}


\begin{definition}\label{energy_decrease}
\lean{Hopfield82.energy_decrease}
\uses{PhaseSpacePoint}
\leanok
The `energy\_decrease` when updating neuron i is always non-positive,
as proven in the paper with Equation 8.

This theorem formalizes a key result from the paper: the energy function
always decreases (or remains constant) under asynchronous updates.
\end{definition}


\begin{definition}\label{convergence_to_fixed_point}
\lean{Hopfield82.convergence_to_fixed_point}
\uses{HopfieldNet_convergence_fair}
\leanok
This theorem captures the convergence result from the paper:
"Every initial state flows to a limit point (if synchrony is not assumed)."
\end{definition}

\section{Memory Storage Algorithm}

\begin{definition}\label{normalizedPattern}
\lean{Hopfield82.normalizedPattern}
\uses{HopfieldNetwork}
\leanok
The `normalizedPattern` converts a neural state to a vector with -1/+1 values,
matching the $(2V_i - 1)$ term from equation 2 in Hopfield's paper.
\end{definition}

\begin{definition}\label{hebbian}
\lean{Hopfield82.hebbian}
\uses{HopfieldNetwork}
\leanok
The `hebbian` function computes the weight matrix according to Equation 2 of Hopfield's paper:
    $T_{ij} = \sum_{s} (2V_{i}^s - 1)(2V_{i}^s - 1) with T_{ii} = 0$
Note that this is equivalent to the existing `Hebbian` definition in HopfieldNet.HN,
but we make the connection to the paper explicit here.
\end{definition}

\begin{definition}\label{isPseudoOrthogonal}
\lean{Hopfield82.isPseudoOrthogonal}
\uses{HopfieldNetwork}
\leanok
The `pseudoOrthogonality` property from Hopfield's paper (Equations 3-4) states:
For random patterns, the dot product between different patterns is approximately 0,
while the dot product of a pattern with itself is approximately N.
This property is essential for understanding the storage capacity of Hopfield networks.
\end{definition}


\section{Connection to Physical Systems}

\begin{definition}\label{spin_glass_analogy}
\lean{Hopfield82.spin_glass_analogy}
\uses{HopfieldNetwork}
The `spin\_glass\_analogy` formalizes the connection between Hopfield networks and
physical spin glass systems, as discussed in the paper.
From the paper (p.2555): "This case is isomorphic with an Ising model."
\end{definition}

\begin{definition}\label{energy_convergence}
\lean{Hopfield82.energy_convergence}
\uses{HopfieldNet_convergence_fair, FixedPoint, PhaseSpacePoint}
The `energy\_convergence` theorem formalizes the connection between energy minimization
and the convergence to fixed points.
This theorem establishes the connection between energy minimization and convergence
to fixed points in Hopfield networks, as described in the paper
From the paper (p.2555): "State changes will continue until a least (local) E is reached."
\end{definition}



% \begin{definition}\label{non_deleted_neuron_maintains_sign_of_activation}
% \uses{deleted_field_product_bound}
% TO DO
% \end{definition}




% \begin{definition}\label{non_deleted_neuron_maintains_sign_of_activation}
% \uses{deleted_field_product_bound}
% TO DO
% \end{definition}




% \begin{definition}\label{non_deleted_neuron_maintains_sign_of_activation}
% \uses{deleted_field_product_bound}
% TO DO
% \end{definition}




% \begin{definition}\label{non_deleted_neuron_maintains_sign_of_activation}
% \uses{deleted_field_product_bound}
% TO DO
% \end{definition}
















\section{Content Addressable Memory}

\begin{definition}\label{retrievalDistance}
\lean{Hopfield82.retrievalDistance}
\uses{PhaseSpacePoint}
\leanok
The `retrievalDistance` function measures how far from a pattern we can initialize
the network and still have it converge to that pattern. 
From the paper (p.2556): "For distance ≤ 5, the nearest state was reached more than
90\% of the time. Beyond that distance, the probability fell off smoothly.
\end{definition}

\begin{definition}\label{ContentAddressableMemory}
\lean{Hopfield82.ContentAddressableMemory}
\uses{HopfieldNet_stabilize}
\leanok
A Content Addressable Memory is a system that can retrieve a complete pattern
from a partial or corrupted version.

This formalizes the central concept from the paper (p.2554):
"A general content-addressable memory would be capable of retrieving this entire
memory item on the basis of sufficient partial information.
\end{definition}

\begin{definition}\label{MetricDecayFunction}
\lean{Hopfield82.MetricDecayFunction}
\leanok
A generic function type representing how a metric (like completion probability or familiarity)
decays or changes based on a distance-like value and network size.
\end{definition}

\begin{definition}\label{ExponentialDecayMetric}
\lean{Hopfield82.ExponentialDecayMetric}
\uses{MetricDecayFunction}
\leanok
A specific exponential decay model, often used to model probabilities or familiarity scores.
This corresponds to `exp(-value / (N/C))` where C is a constant (e.g., 10).
\end{definition}

\begin{definition}\label{AbstractCompletionProbability}
\lean{Hopfield82.AbstractCompletionProbability}
\uses{MetricDecayFunction}
\leanok
The `AbstractCompletionProbability` measures the likelihood of correctly completing a pattern
as a function of the Hamming distance `d` from the stored pattern.
This formalizes the empirical finding from the paper (p.2556):
"For distance $\leq 5,$ the nearest state was reached more than 90\% of the time.
Beyond that distance, the probability fell off smoothly."
\end{definition}

\begin{definition}\label{ErrorCorrection}
\lean{Hopfield82.ErrorCorrection}
\uses{ContentAddressableMemory,HopfieldNet_stabilize}
`ErrorCorrection` quantifies the network's ability to correct errors in the input pattern.
It's measured as the reduction in Hamming distance to the closest stored pattern after convergence.
\leanok
\end{definition}

\begin{definition}\label{error_correction_guarantee}
\lean{Hopfield82.error_correction_guarantee}
\uses{ContentAddressableMemory,HopfieldNet_stabilize}
\leanok
The `error\_correction\_guarantee` theorem establishes that Hopfield networks
can correct a substantial fraction of errors in the input pattern.

This formalizes a key capability of content-addressable memories.
\end{definition}

\begin{definition}\label{BasinOfAttraction'}
\lean{Hopfield82.BasinOfAttraction'}
\uses{ContentAddressableMemory,HopfieldNet_stabilize}
\leanok
The `BasinOfAttraction'` of a pattern is the set of all states that converge to it.
This concept is central to understanding the storage and retrieval properties of Hopfield networks.

From the paper (p.2554): "Then, if the system is started sufficiently near any Xa,
as at $X = Xa + \Delta,$ it will proceed in time until $X \equiv Xa.$
\end{definition}

\begin{definition}\label{BasinVolume}
\lean{Hopfield82.BasinVolume}
\uses{ContentAddressableMemory,HopfieldNet_stabilize}
\leanok
The `BasinVolume` is the "size" of the basin of attraction, measured as the
fraction of the state space that converges to a given pattern.

This quantifies the robustness of memory retrieval.
\end{definition}

\begin{theorem}\label{basin_volume_bound}
\lean{Hopfield82.basin_volume_bound}
\uses{ContentAddressableMemory}
\leanok
The `basin\_volume\_bound` theorem establishes that the basin volume decreases
exponentially with the number of stored patterns.
This formalizes how memory capacity affects retrieval robustness.
\end{theorem}

\begin{definition}\label{AbstractFamiliarityMeasure}
\lean{Hopfield82.AbstractFamiliarityMeasure}
\uses{HopfieldNetwork}
\leanok
`AbstractFamiliarityMeasure` quantifies how familiar a given state `s` is to the network,
based on its proximity (closest Hamming distance) to stored patterns, using a provided `decay\_func`.

The paper discusses (p.2557): "The state 00000... is always stable. For a threshold of 0, this
stable state is much higher in energy than the stored memory states and very seldom occurs."
A high familiarity measure (close to 1) indicates `s` is similar to a stored pattern.
The `ExponentialDecayMetric` can be used as `decay\_func`.
\end{definition}


\section{Memory Capacity}

\begin{definition}\label{StorageCapacity}
\lean{Hopfield82.StorageCapacity}
\leanok
The `StorageCapacity` of a Hopfield network is the maximum number of patterns
that can be stored and reliably retrieved. The paper suggests this is around 0.15N.
From the paper (p.2556): "About 0.15 N states can be simultaneously remembered before
error in recall is severe."
\end{definition}

\begin{definition}\label{Hopfield82.Real.erf}
\lean{Hopfield82.Real.erf}
\leanok
The error function is defined as:
  %erf(x) = (2/π) ∫₀ˣ e^(-t²) dt
  \[
\operatorname{erf}(x) = \frac{2}{\sqrt{\pi}} \int_0^x e^{-t^2} \, dt
\]
This function is central in probability theory, especially for normal distributions.
\end{definition}

\begin{definition}\label{PatternRetrievalError}
\lean{Hopfield82.PatternRetrievalError}
\uses{Hopfield82.Real.erf}
\leanok
The `PatternRetrievalError` function computes the probability of an error in pattern retrieval
for a network storing m patterns. This increases as m approaches and exceeds 0.15N.
This corresponds to the error probability P discussed in Equation 10 of the paper:
\( P = \int_{\sigma}^{\infty} \frac{1}{\sqrt{2\pi}} e^{-x^2 / 2} \, 
dx = \frac{1}{2} \left(1 - \operatorname{erf}\left(\frac{\sigma}{\sqrt{2}}\right)\right) \),
 where \( \sigma = \frac{N}{2\sqrt{nN}} \) and \( n \) is the number of patterns.

\end{definition}

\begin{theorem}\label{storage_capacity_bound}
\lean{Hopfield82.storage_capacity_bound}
\uses{HopfieldNetwork, Hebbian_stable}
\leanok
The result from the paper that a Hopfield network can store approximately 0.15N patterns
with high reliability, where $N$ is the number of neurons. This theorem formalizes the
 key result about storage capacity from the paper,
utilizing the Hebbian\_stable theorem from the existing codebase.
\end{theorem}


\section{Fault Tolerance}

\begin{definition}\label{DeleteNeuron}
\lean{Hopfield82.DeleteNeuron}
\uses{HopfieldNetwork}
\leanok
The `DeleteNeuron` function simulates the failure of a neuron by removing its connections.
This corresponds to setting weights to/from that neuron to zero.
The paper discusses (p.2558): "The collective properties are only weakly sensitive to
details of the modeling or the failure of individual devices."
\end{definition}

\begin{definition}\label{FaultTolerance}
\lean{Hopfield82.FaultTolerance}
\uses{DeleteNeuron}
\leanok
The `FaultTolerance` of a Hopfield network is its ability to maintain function 
despite the failure of some components. The paper notes that these networks are 
inherently robust to component failures.
\end{definition}


\begin{definition}\label{pattern_stability_in_hebbian}
\lean{Hopfield82.pattern_stability_in_hebbian}
\uses{Hebbian_stable}
\leanok
When m is at most a tenth of total neurons, each pattern is fixed point in the undamaged network
\end{definition}


\begin{lemma}\label{delete_single_neuron_step}
\lean{Hopfield82.delete_single_neuron_step}
\uses{DeleteNeuron,hebbian}
\leanok
When deleting a single neuron from a network, the resulting weighted sum for a neuron u
that's not the deleted neuron equals the original weighted sum minus the contribution
from the deleted neuron.
\end{lemma}

\begin{definition}\label{delete_empty_neurons_step}
\lean{Hopfield82.delete_empty_neurons_step}
\uses{DeleteNeuron,hebbian}
\leanok
When deleting neurons from an empty list, the result is the original network
\end{definition}

\begin{definition}\label{delete_cons_neuron_step}
\lean{Hopfield82.delete_cons_neuron_step}
\uses{DeleteNeuron}
\leanok
When deleting a list of neurons with a new neuron added at the front, the effect 
on the weighted sum equals the effect of deleting the first neuron and then 
deleting the rest of the list.
\end{definition}

\begin{definition}\label{delete_singleton_neuron_step}
\lean{Hopfield82.delete_singleton_neuron_step}
For a singleton list, the effect matches the single neuron deletion case.
\leanok
\end{definition}

\begin{definition}\label{delete_neuron_from_deleted_network}
\lean{Hopfield82.delete_neuron_from_deleted_network}
\uses{Hopfield82.Real.erf}
\leanok
The effect of deleting a neuron from an already deleted network on a neuron u that
is not in the deleted set.
\end{definition}

\begin{definition}\label{commute_delete_foldl}
\lean{Hopfield82.commute_delete_foldl}
\uses{DeleteNeuron}
\leanok
Helper lemma: DeleteNeuron commutes with foldl of DeleteNeuron if the neuron is not in the list.
\end{definition}


\begin{definition}\label{foldl_delete_preserves_weights}
\lean{Hopfield82.foldl_delete_preserves_weights}
\leanok
Helper lemma: Weights are preserved by foldl if indices are not in the list
\end{definition}

\begin{definition}\label{delete_neurons_recursive}
\lean{Hopfield82.delete_neurons_recursive}
\uses{foldl_delete_preserves_weights}
\leanok
Helper lemma: Deleting a list of neurons recursively subtracts their contributions.
    Requires that the list of neurons has no duplicates.
\end{definition}

\begin{definition}\label{deleted_neurons_field_effect}
\lean{Hopfield82.deleted_neurons_field_effect}
\uses{delete_neurons_recursive}
\leanok
DeleteNeurons removes weights connected to deleted neurons.
\end{definition}

\section{Lemmas for Bounding Field Reduction due to Neuron Deletion}

\begin{definition}\label{hebbian_weight_deleted_neurons_l_eq_k_term}
\lean{Hopfield82.hebbian_weight_deleted_neurons_l_eq_k_term}
\uses{Hopfield82.Real.erf}
\leanok
Calculates the contribution to the deleted field sum from the target pattern `k` itself
in the Hebbian weight definition.
\end{definition}

\begin{definition}\label{hebbian_weight_deleted_neurons_cross_talk_term}
\lean{Hopfield82.hebbian_weight_deleted_neurons_cross_talk_term}
\uses{Hopfield82.Real.erf}
\leanok
Defines the cross-talk contribution to the deleted field sum.
This term arises from the interaction of the target pattern `k` with other stored patterns $`l \neq k`$
over the set of deleted neurons.
\end{definition}

\begin{lemma}\label{cross_talk_term_abs_bound_assumption}
\lean{Hopfield82.cross_talk_term_abs_bound_assumption}
\uses{hebbian_weight_deleted_neurons_cross_talk_term}
Axiom stating the bound on the absolute value of the cross-talk term.
This encapsulates the statistical argument from Hopfield's paper that for
random-like patterns, the sum of interfering terms is bounded.
\end{lemma}

\begin{lemma}\label{Hebbian_stable}
\lean{Hopfield82.Hebbian_stable}
\uses{hebbian}
\leanok
Decomposes the total sum representing the field reduction into the contribution
from the target pattern `k` and the cross-talk term from other patterns $l \neq k.$

**Hopfield Assumption:** Assumes the standard Hebbian learning rule where $T_{ii} = 0$.
The `Hebbian` definition in `HN.lean` implements this by subtracting `m • 1`.
\end{lemma}

\begin{lemma}\label{bound_cross_talk_term}
\lean{Hopfield82.bound_cross_talk_term}
\uses{hebbian_weight_deleted_neurons_cross_talk_term}
Placeholder lemma for bounding the cross-talk term.
Proving a tight bound likely requires assumptions beyond simple orthogonality,
such as patterns being random and uncorrelated, analyzed in the limit of large N.
The bound likely depends on `m` (number of patterns) and `N` (number of neurons).
**Hopfield Assumption:** Implicitly assumes patterns behave statistically like random vectors.
\end{lemma}

\begin{lemma}\label{deleted_field_bound}
\lean{Hopfield82.deleted_field_bound}
\uses{bound_cross_talk_term}
The field reduction from deleting neurons has a bounded effect.
This version uses the decomposition and the (unproven) cross-talk bound.
\end{lemma}

\begin{lemma}\label{field_remains_sufficient}
\lean{Hopfield82.field_remains_sufficient}
\leanok
With constrained m and limited deleted neurons, the field remains strong enough
\end{lemma}

\begin{lemma}\label{bound_cross_talk_term_abs}
\lean{Hopfield82.bound_cross_talk_term_abs}
\uses{cross_talk_term_abs_bound_assumption}
\leanok
For random orthogonal patterns, the cross-talk term has a bounded absolute value.
This is a fundamental assumption from Hopfield's paper about how patterns interact.
\end{lemma}

\begin{lemma}\label{deleted_field_product_bound}
\lean{Hopfield82.deleted_field_product_bound}
\uses{hebbian_weight_deleted_neurons_cross_talk_term}
TO DO
\end{lemma}

\begin{lemma}\label{field_remains_sufficient_for_N_div_5}
\lean{Hopfield82.field_remains_sufficient_for_N_div_5}
\uses{Hopfield82.Real.erf}
With bounded numbers of patterns and deleted neurons, the field remains strong enough
    to maintain the pattern stability, adjusted for N/5 bound.
\end{lemma}

\begin{lemma}\label{hebbian_deleted_threshold_is_zero}
\lean{Hopfield82.hebbian_deleted_threshold_is_zero}
\uses{DeleteNeuron}
TO DO
\end{lemma}

\begin{lemma}\label{net_input_at_non_deleted_neuron}
\lean{Hopfield82.net_input_at_non_deleted_neuron}
\uses{deleted_neurons_field_effect}
TO DO
\end{lemma}

\begin{lemma}\label{product_net_input_activation_at_non_deleted_neuron}
\lean{Hopfield82.product_net_input_activation_at_non_deleted_neuron}
\uses{net_input_at_non_deleted_neuron}
TO DO
\end{lemma}

\begin{definition}\label{non_deleted_neuron_maintains_sign_of_activation}
\lean{Hopfield82.non_deleted_neuron_maintains_sign_of_activation}
\uses{deleted_field_product_bound}
TO DO
\end{definition}

\begin{definition}\label{DeleteNeurons_with_Finset}
\lean{Hopfield82.DeleteNeurons_with_Finset}
\uses{DeleteNeuron}
When deleting neurons from a Finset, we can use Finset.toList to convert the Finset to a List.
This matches the API needed by DeleteNeurons.
\end{definition}

\begin{definition}\label{fault_tolerance_bound}
\lean{Hopfield82.fault_tolerance_bound}
\uses{deleted_field_product_bound}
\leanok
A Hopfield network can tolerate the failure of up to 10\% of its neurons
    while maintaining all stored patterns as fixed points, provided:
\begin{enumerate}
\item The stored patterns are orthogonal
\item The number of patterns is at most 10\% of the network size
\end{enumerate}
\end{definition}
